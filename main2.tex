\documentclass{article}
\usepackage{fullpage}
\usepackage{setspace}
\usepackage{changepage}
\usepackage[perpage]{footmisc}
\begin{document}

%%%%%%%%%%%%%%%%%%%%%%%%%%%%
% Macros
%%%%%%%%%%%%%%%%%%%%%%%%%%%%

% Centered horizontal line
\newcommand{\choar}{%
\begin{center}
\line(1,0){350}
\end{center}
}

% Left aligned/default aligned horizontal line
\newcommand{\lhoar}{%
\noindent\line(1,0){200}
}

% Really so I can switch between single and
% double space easily to save paper for printing.
% \revertspace tells it to go back to whatever it's overwritten to.
% Default is single spaced.
% \doublenewpage is to add a newpage for doublespace mode.
% \singlenewpage adds a newpage for singlespace mode.
\newcommand{\revertspace}{\singlespace}

\newcommand{\doublenewpage}{}
\newcommand{\singlenewpage}{\newpage}

\newcommand{\setdoublespace}{%
  \doublespace
  \renewcommand{\doublenewpage}{\newpage}
  \renewcommand{\singlenewpage}{}
  \renewcommand{\revertspace}{\doublespace}
}

\newcommand{\setsinglespace}{%
  \singlespace
  \renewcommand{\doublenewpage}{}
  \renewcommand{\singlenewpage}{\newpage}
  \renewcommand{\revertspace}{\singlespace}
}



%%%%%%%%%%%%%%%%%%%%%%%%%%%%
% Spacing macros
%%%%%%%%%%%%%%%%%%%%%%%%%%%%

% Levels of vertical space
\newcommand{\vv}{\vspace*{1ex}}
\newcommand{\VV}{\vspace*{2ex}}
\newcommand{\vvvv}{\vspace*{8ex}}

% Levels of horizontal space
\newcommand{\hh}{\hspace*{2ex}}
\newcommand{\HH}{\hspace*{4ex}}
\newcommand{\hhhh}{\hspace*{8ex}}


%%%%%%%%%%%%%%%%%%%%%%%%%%%%
% Combined macros
%%%%%%%%%%%%%%%%%%%%%%%%%%%%

% Left Horizontal line + line break.
\newcommand{\lhoarb}{\\\lhoar\VV\\}
\newcommand{\lhoarnob}{\lhoar\VV\\}

% Header + Left horizontal line.
\newcommand{\dayheader}[1]{%
  \singlespace\noindent\large\textbf{#1}\normalsize\\
  \lhoar\revertspace
  \VV
}
\newcommand{\dateheader}[1]{%
  \singlespace\noindent\textbf{#1}\\
  \lhoar\\\revertspace
}

%%%%%%%%%%%%%%%%%%%%%%%%%%%%
% Character names
%%%%%%%%%%%%%%%%%%%%%%%%%%%%

% Names are hard, macros for easy changing.

% Main Story Characters
\newcommand{\emma}{Emma}
\newcommand{\eric}{Eric}
\newcommand{\april}{Avril}
\newcommand{\josh}{Quaid}
\newcommand{\dameon}{Dameon}
\newcommand{\jasmine}{Jasmine}
\newcommand{\mike}{Mike}
\newcommand{\crosby}{James}
\newcommand{\casey}{Casey}
\newcommand{\blake}{Blake}

% Main Story Places
\newcommand{\cafe}{caf{\'e}}
\newcommand{\cafeF}{Caf{\'e} Breleir}


% Secondary Story Characters
\newcommand{\ugly}{Ugly}
\newcommand{\uglyF}{\ugly{} the Unicorn}
\newcommand{\yolo}{Yolo}
\newcommand{\yoloF}{\yolo{} the Yodling Billygoat}
\newcommand{\pixel}{Pixel}
\newcommand{\pixelF}{\pixel{} the Cat}

% Secondary Story Animals
\newcommand{\peacock}{Prassilla}
\newcommand{\peacockF}{\peacock{} the Peacock}
\newcommand{\goose}{Garilda}
\newcommand{\gooseF}{\goose{} the Goose}
\newcommand{\dove}{Regulus}
\newcommand{\doveF}{\dove{} the Dove}
\newcommand{\pidgen}{Alphard}
\newcommand{\pidgenF}{\pidgen{} the Pidgen}

%%%%%%%%%%%%%%%%%%%%%%%%%%%%
% Settings
%%%%%%%%%%%%%%%%%%%%%%%%%%%%

% The following should probably be in the mains instead.

\interfootnotelinepenalty=10000
\raggedbottom
\addtolength{\topskip}{0pt plus 10pt}

\setdoublespace
%\setsinglespace


\begin{center}
\large\textbf{I Dreamt of Death}
\end{center}

% Dread should be a character.
% personify dread everywhere there is fear.

% Avoid contractions
% "Quilifiers are the leeches that infest the pond of prose,
%  sucking the blood of words" (clearly, many, really)
% Avoid using 'certainly'
% Prefer positive form instead of negative of sentences (not everywhere)
% Avoid 'would', it weakens the sentence

\noindent
Dread.
Dread is beautiful--he is salacious--he is engrossing--he
gives reason to action and inaction.
Dread captivates the mind and disheartens and depresses
those gripped by his claws.
And he has sister, Fear--his dear little sister.
She is shorter.  She is sweeter than he.
While Dread lingers and attaches parasitic,
Fear comes and goes as she pleases--leaving
little jitters in her wake.


I have always been fond of the dark and of terror--what
lurks inside the minds of men--since in my years
I have looked--I have seen the depths of Hell
and gleamed a glimmer of droll hostility
inhabiting the souls of those we call neighbors.
Is this why I find myself in question--constant question--by
the youth of this day as to my sanity?
I urge reconsideration of this preposterous proposition that
I, myself, am mad.
This world is unstable and, I, in time, a discrete blip--a
dull idea of light--lost to all who succeed my passing.
I am, most certainly, \textit{not} mad.
\VV


\noindent % reword this shit
Certainly it is time to tell my tell--the
loss of my innocence that I had so longed to keep--the
source of my dread.
%Since the innocence I so longed to keep has fled me so,
%I have no holds to tell this tale. %I have held close and closed.
\VV


\noindent
Years ago, on All Hollow's Eve,
I met the devil and she met me.
\VV


\noindent
In a manner, it was a party like any other.
Habitually incoherent by whiskey,
we all danced and laughed and reminisced of festivities past.
Half through the night, I spotted the dame,
vivacious and blond and standing alone with a beer.
And so I found myself there, next to the beauty,
exploring the corners of my sagacity with a horrid
grin and appalling optimism.
In memory I am certain that she was a friend of
a friend of a friend; to what end, I am unsure.
The details of that night are not entirely clear.
All I knew were those alluring eyes were drawing me in.
Captivated, entranced, and intoxicated,
I took the chance; the beer saw to that.
I thought it was fate for it is unusual
that good follows such in-eloquent words
as mine in stupor.
But there they were, anomalous sounds,
silver words from a queen observing each
locality of my wildest dreams.


With the night fulfilled, we parted ways.
I waited in lust for days and days.
Only a week lay between then and our meeting,
but the hours were too long for my perplexed anxiety;
and with my want came the first sign that this woman may not be divine.


It was not my captivation.
It was not my intoxication.
It was the poison of thoughts.
The grand immaculation of the mind.
A spark of dread held so closely predestinate: dreams.
\VV


\noindent
Dreams are a world and they mirror our own.
They speak to our tears, our joys, and our fears.
I, at the time, was oblivious that the scenes they
showed were so linked to reality.


Perhaps I would have known had I listened--really listened--to
what Castaneda had to say.
He often spoke of Ixtlan,
a separate reality,
psychotropic in form,
a lucid plane on which he learned of the knowledge dreams hide.
I thought him mad, but now no longer;
for the same scrutinous eyes that looked so down upon him
have turned to me.
\VV


\noindent
The first of my dreams was subtle, a test, perhaps
for the waters were still murky and unclear.
\vvvv


\noindent
I sat in a tent with an eld' hag my opposite.
She begged me to drink the sands from a misplaced hourglass.


Bound in silver, the fragile glass construction had
carved upon the sheath a myriad of demons.
Red sand--carmine--fell within.


Obediently I partook.


\noindent
I was in darkness, but a flame bobbed closer from a distance.


Decaying skin and ebony eyes peered at me from behind the light.
Crouching, disheveled, and doleful.
The more I looked, the more it stared.
And the more it stared, the more it dawned.
The creature was me and I was he.
We differed by skin and cloth; but our bones were one
and our eyes spoke the same.


The other me hobbled off down unknown ways.
I followed close over the wet of the fungi that
covered the cavity floor.
The passage stretched on for what seemed like miles.
I tired and slowed as the light become a distant memory.
But I continued onward, following the walls with my hands.


I miss-turned in the black and my feet met an edge.
With my senses dulled by isolation, I responded too late.
The greeting was over and this fall my fate.
\vvvv


\noindent % reword
The day came--a fantasy at first.
I could think of none better. % 'could' weakens the sentence
She took me through the woods wherein laid a garden--a
secret--a haven--the hidden refuge where
the last of the bloodflowers grew.
We walked and we walked for what seemed like hours
while she told me stories of her former home.
I fell fascinated--intrigued--by the lips
upon which those words came--supple and pale.


With our heads lain against a tree we
fell into the languid tranquility of the garden.
I closed my eyes and let the cool breeze envelope me
and urge me to sleep. All was quiet and I at peace.


I awoke in pain.
A force--a hand--wrapped around my throat--until I screamed.
That whore of a woman had her hands on me!
Confused and in fear I laid still
with her grinning ear to ear.
Quickly! How quickly did that expression change!
Once a smirk and then concern.
``I am sorry'' she pleaded, her tone never sweeter.
Oh how I regret my stupidity!
How I curse my love-tainted heart!
For despite that feeling--that
tyrannical grip--that lingered on my larynx--I
caved and forgave that iridescent witch.


That night I dreamt again; a dream more suspicious;
a dream more fanatical but ever more clear.
A fear of that woman who grinned ear to ear. % Repetition bad.
\vvvv


\noindent
Deep in the forest--a clearing--where a mansion lay.
Luxurious--it was--unimaginably so.
Immaculate mosaics--stained in glass--flanked the walls.
Inside--there she stood, draped in a soft violet nightgown,
adorning the balcony overlooking the foyer.


A pentacle star--vermilion--ornate--obscured the floor.
She stood without trace of lunacy--without
maniacal ambition.
I do not know of what, but her eyes--her eyes
shown with a want so embolden to drive
such incredulous lengths.
They concentrated on the star--not
glazed--not soft, but astute; and with
that focus she articulated a hideous prayer.
\VV


% Ritual... Latin... need more
\singlespace
\begin{tabular}{ll}
& \textit{Oremus} \\
& \textit{Grandis Diabolus} \\
& \textit{Elevare legionis} \\
& \\
& \textit{Oremus} \\
& \textit{E nomine} \\
& \textit{Episcopus Malificus} \\
& \\
& \textit{Oremus} \\
& \textit{Educet dolorem} \\
& \textit{Et sumere ultio} \\
\end{tabular}\revertspace\footnote{
Translation:
Let us pray // Great Devil // Raise legions //
Let us pray // From name // Bishop of black magic //
Let us pray // (He) brings pain // And exacts revenge \\
\indent
Note: Declensions are little off for flow.
} \vspace*{3ex}


\noindent % Shity paragraph
The grand redwood doors guarding the mansion burst in
before she completed the ritual.
I did not see the nature of the beast who so defiled the
beauty of those doors.
She ran, in search of escape.
She hurled a doily-laden table through the largest
of the mosaics and leapt through the opening
into the lake below.


Her nightgown hugged tight to her licentious body
as she pulled herself from the water.
Therefound a lone hermit--haggard--unplaced.


``The Black Bishop on whom thou hast tempted
with blood seeks thou his bride.''
\vvvv


\noindent
The dream--short, but to a point--solidified my thoughts as dread.
Still, I reserved myself from such atrocity.
For the solemn hours I spent in solitude through the nights
overcame my judgment.
Such a frail girl.
What had I to fear?
The glow from her silken blond hair in evening light?
The pale complexion of her skin, almost translucent at times?
Her chin--rounded in such adorable symmetry?
Or was it those eyes? Those endearing green eyes?
Those eyes for which countless nations might rise and fall?
I had fallen for that succubus.
She had stolen my heart and my groin;
and my brain no longer won.
\VV


\noindent
I may be a romantic; but I urge you, I am no fool.
I personally planned the second of our dates--a public venue--a symphony.


We arrived separately; I saw to that.
With no time to socialize we swiftly found our seats.
The concert hall held a certain radiance--an elegance
so vivid--so unadorned.
The orchestra, all with brass or wood, seamlessly--flawlessly--followed
the maestro's hypnotic hands.


Oh, how quickly it concluded; and how quickly the night turned.
We talked of the music--those enthralling notes.
Her voice was smooth--it soothed my soul--it purified
the damned--the dread in my gut.
Before long, I had forgotten myself.


I brought her home; a horrible mistake.
We laid together in my unkempt room.
How perfect it seemed--how
absolutely perfect--the
softness of her skin--that
low moaning sigh.
I held her tight, not wishing for an end.
\VV


\noindent
With my lust exhausted--my stamina drained--I fell unconscious.
\VV


\noindent
It might have been the nails driven into my exposed abdomen
or it might have been the sadistic titillation on her face,
but I finally had suspect of foul-play.
I threw the bitch off, wincing in pain.
I ordered her out and she followed suit.
She left so obediently--so suspiciously--at
the time, however, the nails concerned me most.


I wondered long if she remembered my address.
With the sanctity of my home tainted, I needed to move.
I spent the day packing and nursing my wound.
No time to leave; I simply boarded my home--every
window--every door--every vent upon the floor--no
crevice left exposed.
I did not care if I might suffocate on stale air--if
it meant her unable to choke me first.


My anxiety grew and grew, but still I slept
with adulation for my sagacious sealing.
\vvvv


\noindent
I fell through the clouds onto a phosphorescent floor.
I found myself--amidst the haze of light--affront
a prodigious dragon clothed by lamella\footnote{Scales.}. % confuzzling?
It spoke, gray smoke rising from its snout with every breath.
\VV


``In the hall of prophecy you will find your home.''
\VV


\noindent
Great stones arose from the radiant white marble floors.
Glyphs of unknown languages carved into each.
I wandered up and down those halls of stone
until I spied one that seemed familiar.
It is hard to describe; but it felt like a long lost friend
reunited.  I read aloud the words codified for eternities.
\VV


%\begin{tabular}{ll}
%& Long buried \\
%& beneath the island crust \\
%& Where points the fingers \\
%& of the three-leafed tree; \\
%& Where sleeps the flees \\
%& in countless droves; \\
%& Where nests the ostrich \\
%& outside itself. \\
%\end{tabular}
%\vspace*{3ex}


\noindent
The glow from the floor grew brighter;
enveloping me in blinding white.
The light dulled and
all around me were the docks of a poor fishing village.
\VV


``I am in search of an island.''
I proclaimed to a man readying his boat.
\VV


\noindent
He looked at me, crestfallen.
\VV


``Only one island 'round here.
I can take you,'' he said, ``for your shoes.''
\VV


\noindent
I had no use for shoes; and he offered to help my journey.
So I gave them to the man and we sailed off.


On the island, I met her; she had the ancient shovel.
Together we found the three-leafed tree, the
wayward ostrich, and the graveyard of flees.


I brandished the shovel and thrust it into the dirt
where the fingered tree pointed--where the ostrich
chose to sit--where the bodies of the flees lay
countless--encrusting the Earth.


I dug for hours and hours; focused solely on the task.
Finally, I broke through; my home below--the
light on my nightstand shining beneath me.


A sharp pain erupted from the back of my skull.
She grunted and pulled, yanking my head with her.
The pressure left me and I felt a coolness in
places that have never felt the wind before.
And then again, another cleave, this time to my brain.
I fell into the hole; I had dug my grave.
\vvvv


\noindent
Dread had ensnarled me--had seduced me.
He whispered horrors to me--terrors
more heinous--more \textit{dread}ful.
He impregnated my mind with his salacious words and
those of Mephistopheles rang out in my head.
\VV


\textit{For this is Hell, nor am I out of it.}
\VV


\noindent
But may I raise such demons?
May I commit such evil of body and of soul?
How am I deluged with such hostility towards fellow man?
I must take that fiend's sullied heart;
for without I am ever unsafe.
\VV


\noindent
I invited her out once more to that garden--that
garden so hidden--so removed from the world--where
few ever go. % reword. no 'few'


She accepted my offer.
She had no reason to refuse.
And so I met with her outside the woods.


She had that stupid grin on her face.
Always grinning. Always scheming.
It still sickens me to this day.


I greeted her as usual and led her into the trees.


We stopped for a while, before crossing the landslide.
There resided a sinkhole.
A great gaping hole into unknown depths.
A fence surrounded it,
but I leaned over just a bit.
I wanted to see that darkness--that
blackest of blacks.
And in that gloom and nothingness I thought, for a moment,
that I saw a light; a soft flickering from a candle.


My mind became hazy with each step further we took.
I took precaution to keep up conversation with
my usual demeanor.
I found it more and more difficult to remain at ease the closer we became.


At last we arrived in the garden.
The bloodflowers had reached full bloom.
I wavered.  I did not want to defile the beauty--the serenity.
But I found myself again.


I held her to my side as we looked up at the moon.
Slowly--very slowly--very cautiously--I
detached the hatched from its sheath on my belt--not making a sound.


She turned and looked at me; her green eyes shimmering in the twilight.
I pushed her down with my hand around her throat.
Those eyes never left my own. They penetrated me.
She knew her fate.


I flipped her over, pinning her arms with my legs.
I drove the hatchet into her neck--over and over.
Blood speckled my face and my clothes but I kept going
until I sundered the head from that succubus.
Her head rolled off with the final strike.
It stared at me; its eyes still open.


I drove home that night;
delighted to take down the boards.
I had nothing left to fear.
\VV


\noindent
The next day a couple of hikers found her body.
They closed off the garden and the trail
urging everyone to stay away.


A months time went before the funeral came.
I felt compelled to attend.


I spoke at her funeral--of her beauty that I loved.
I hugged her family.
I gave condolences to her friends.
No one suspected her blood had been on my hands.


They had sewn her head back on and covered the
stitching with a white choker.
She looked lovely in white.
She looked lovely in everything.
\VV


\noindent
Time passed and the investigation slowed.
I had not slept since the funeral.
Every time I closed my eyes I saw her stricken expression--heard
her dreadful screams--as I hacked away her head.


I refused to give in and I began to preach on the power of dreams.
I have written pamphlets and letters and books;
all on the warnings and the cautions they tell.
Each time--each lecture--brought me some peace
until, finally, I managed to sleep again.
\vvvv


\noindent
I was standing in a park reminiscent of
Georges Seurat's ``A Sunday Afternoon on the Island of La Grande Jatte''.
Countless people looked out onto the crystal clear pond.
They were faceless yet unique.
A woman sat on her knees, shaded by a parasol, picking flowers.
She wore a long pink skirt with an orange top and hat.
Dogs wandered around happily.
A couple of men were relaxing by a tree smoking pipes.
The tranquility of the park had absorbed me.


The sky turned a dark and hideous red that bled into the world;
staining all.  Hooded figures stormed the area, killing everyone in sight.
I ran, avoiding the fallen bodies, to the water.
I thought I might be able to hide under the surface or at least
reach the opposite shore.


I stopped just short of the lake's edge and gazed out over the bloody water.
The surface bubbled--molten.
A great black spire erupted from the water.
The imposing figure of the tower seemed to look down on the world--to
look down upon me.
I dropped to my knees;
footsteps in slow strides echoed in the hollow air behind me.
\VV


``You cannot run from fate.''
\VV


\noindent
The voice was taunting and familiar.
It was my own.
There was a click and my body dropped.


Again I laid dead.
\vvvv


\noindent
With this my final work,
I hope my lessons have helped some believe in the power of dreams.
It is now time for me to rest and join the evil that has plagued me.
For there is no longer way for me to escape my fate in Hell.
\VV

\begin{center}
\textit{Olim nos conventus super terra.}
\textit{Nunc nos conventus simul apud inferno.}\footnote{
Translation:  Once we met on Earth. Now we meet in Hell. \\
\indent
Note: Latin is a bit funky.  I think I got my conjugations right.
}
\end{center}

\end{document}
