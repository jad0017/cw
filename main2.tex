\documentclass{article}
\usepackage{fullpage}
\usepackage{setspace}
\usepackage{changepage}
\usepackage[perpage]{footmisc}
\begin{document}

\input{macros}

\begin{center}
\large\textbf{I Dreamt of Death and How She Would Kill Me}
\end{center}


\noindent
Dread.
Dread is a beautiful word, though not entirely perfect.
Fear is much better--it's shorter--it's sweeter.
I've always been fond of the dark and of terror--what
lurks inside the minds of men--since in my years
I've looked with words at the depths of Hell
and gleamed a glimmer of droll hostility
inhabiting the souls of those we call neighbors.
Is this why I find myself in question--constant question--by
the youth of this day as to my sanity?
I urge reconsideration of this preposterous proposition that
I, myself, am mad.
This world is unstable and, I, in time, a discrete blip,
a dull idea of light, lost to all who succeed my passing.
I am, most certainly, \textit{not} mad.
\VV


\noindent
Since the innocence I so longed to keep has fled me so,
I have no holds to tell this tale I've held close and closed.
\VV


\noindent
Many years ago, on All Hollow's Eve,
I met the devil and she met me.
\VV


\noindent
In a manner, it was a party like any other.
Habitually incoherent by whiskey,
we all danced and laughed and reminisced of festivities past.
Half through the night, I spotted the dame,
vivacious and blond and standing alone with a beer.
And so I found myself there, next to the beauty,
exploring the corners of my sagacity with a horrid
grin an appalling optimism.
In memory I'm certain that she had a friend of
a friend of a friend; to what end, I'm unsure.
The details of the night were not entirely clear.
All I knew were those alluring eyes were drawing me in.
Captivated, entranced, and clearly intoxicated,
I took the chance; the beer saw to that.
I thought it was fate for it's unusual
that a date follows such in-eloquent words
as mine in stupor.
But there they were, anomalous sounds,
silver words from a queen observing each
locality of my wildest dreams.


The night fulfilled, we parted ways.
I waited in lust for days and days.
It was just a week before we were set to meet,
but the hours were too long for my perplexing anxiety.
Though with my want came the first sign that this woman may not be divine.


It was not my captivation.
It was not my intoxication.
It was the poison of thoughts.
The grand immaculation of the mind.
A spark of dread held so closely predestinate: dreams.
\VV


\noindent
Dreams are a world and they mirror our own.
They speak to our tears, our joys, and our fears.
I, at the time, was oblivious that the scenes they
showed were so linked to reality.


Perhaps I'd have known had I listened--really listened--to
what Castaneda had to say.
He often spoke if Ixtlan,
a separate reality,
psychotropic in form,
a lucid plain on which he learned the knowledge dreams hide.
I thought him mad, but now no longer;
for now my ideas are faced by the same scrutinous eyes
that looked so down upon him.
\VV


\noindent
The first of my dreams was subtle, a test, perhaps
for the waters were still murky and unclear.

\end{document}
